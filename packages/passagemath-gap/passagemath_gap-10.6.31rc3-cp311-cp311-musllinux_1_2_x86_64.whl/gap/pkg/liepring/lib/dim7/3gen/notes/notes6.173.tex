
\documentclass[12pt]{article}
%%%%%%%%%%%%%%%%%%%%%%%%%%%%%%%%%%%%%%%%%%%%%%%%%%%%%%%%%%%%%%%%%%%%%%%%%%%%%%%%%%%%%%%%%%%%%%%%%%%%%%%%%%%%%%%%%%%%%%%%%%%%%%%%%%%%%%%%%%%%%%%%%%%%%%%%%%%%%%%%%%%%%%%%%%%%%%%%%%%%%%%%%%%%%%%%%%%%%%%%%%%%%%%%%%%%%%%%%%%%%%%%%%%%%%%%%%%%%%%%%%%%%%%%%%%%
\usepackage{amsfonts}
\usepackage{amssymb}
\usepackage{sw20elba}

%TCIDATA{OutputFilter=LATEX.DLL}
%TCIDATA{Version=5.50.0.2890}
%TCIDATA{<META NAME="SaveForMode" CONTENT="1">}
%TCIDATA{BibliographyScheme=Manual}
%TCIDATA{Created=Sunday, July 21, 2013 14:13:46}
%TCIDATA{LastRevised=Sunday, July 21, 2013 17:54:08}
%TCIDATA{<META NAME="GraphicsSave" CONTENT="32">}
%TCIDATA{<META NAME="DocumentShell" CONTENT="Articles\SW\mrvl">}
%TCIDATA{CSTFile=LaTeX article (bright).cst}

\newtheorem{theorem}{Theorem}
\newtheorem{axiom}[theorem]{Axiom}
\newtheorem{claim}[theorem]{Claim}
\newtheorem{conjecture}[theorem]{Conjecture}
\newtheorem{corollary}[theorem]{Corollary}
\newtheorem{definition}[theorem]{Definition}
\newtheorem{example}[theorem]{Example}
\newtheorem{exercise}[theorem]{Exercise}
\newtheorem{lemma}[theorem]{Lemma}
\newtheorem{notation}[theorem]{Notation}
\newtheorem{problem}[theorem]{Problem}
\newtheorem{proposition}[theorem]{Proposition}
\newtheorem{remark}[theorem]{Remark}
\newtheorem{solution}[theorem]{Solution}
\newtheorem{summary}[theorem]{Summary}
\newenvironment{proof}[1][Proof]{\noindent\textbf{#1.} }{{\hfill $\Box$ \\}}
\input{tcilatex}
\addtolength{\textheight}{30pt}

\begin{document}

\title{Algebra 6.173}
\author{Michael Vaughan-Lee}
\date{July 2013}
\maketitle

Algebra 6.173 has presentation 
\[
\langle a,b,c\,|\,ca-bab,\,cb-\omega baa,\,pa,\,pb,\,pc,\,\text{class }%
3\rangle .
\]

If $L$ is a descendant of 6.173 of order $p^{7}$ then the commutator
structure of $L$ is the same as that of one of the $p+2$ algebras with
presentations 7.106 and 7.107 from the list of nilpotent Lie algebras of
dimension 7 over $\mathbb{Z}_{p}$. So we can assume that $L$ has the
following commutator relations 
\[
ca=bab,\,cb=\omega baa,\,baab=\lambda baaa,\,babb=\mu baaa
\]%
for some parameters $\lambda ,\mu $.

If we let $C=\langle c\rangle +L^{2}$ then, if $a^{\prime },b^{\prime
},c^{\prime }$ are the images of $a,b,c$ under an automorphism of $L$, we
have 
\begin{eqnarray*}
a^{\prime } &=&\alpha a+\beta b\func{mod}C, \\
b^{\prime } &=&\pm (\omega \beta a+\alpha b)\func{mod}C, \\
c^{\prime } &=&(\alpha ^{2}-\omega \beta ^{2})c\func{mod}L^{3}
\end{eqnarray*}%
for some $\alpha ,\beta $ which are not both zero.  It follows that 
\begin{eqnarray*}
\lbrack b^{\prime },a^{\prime },a^{\prime },a^{\prime }] &=&\pm (\alpha
^{2}-\omega \beta ^{2})(\alpha ^{2}+2\alpha \beta \lambda +\beta ^{2}\mu
)[b,a,a,a], \\
\lbrack b^{\prime },a^{\prime },a^{\prime },b^{\prime }] &=&(\alpha
^{2}-\omega \beta ^{2})(\omega \alpha \beta +\alpha ^{2}\lambda +\omega
\beta ^{2}\lambda +\alpha \beta \mu )[b,a,a,a], \\
\lbrack b^{\prime },a^{\prime },b^{\prime },b^{\prime }] &=&\pm (\alpha
^{2}-\omega \beta ^{2})(\omega ^{2}\beta ^{2}+2\omega \alpha \beta \lambda
+\alpha ^{2}\mu )[b,a,a,a].
\end{eqnarray*}%
So provided $\alpha ^{2}+2\alpha \beta \lambda +\beta ^{2}\mu \neq 0$ the
effect of this automorphism is to transform the parameters $\lambda ,\mu $ to%
\[
\frac{\pm (\omega \alpha \beta +\alpha ^{2}\lambda +\omega \beta ^{2}\lambda
+\alpha \beta \mu )}{\alpha ^{2}+2\alpha \beta \lambda +\beta ^{2}\mu },\;%
\frac{\omega ^{2}\beta ^{2}+2\omega \alpha \beta \lambda +\alpha ^{2}\mu }{%
\alpha ^{2}+2\alpha \beta \lambda +\beta ^{2}\mu }.
\]%
There are $p+2$ orbits of pairs $\lambda ,\mu $ under this action.

We pick a set representative pairs $\lambda ,\mu $ for these orbits, and get
the following presentations for the descendants of 6.173 of order $p^{7}$:

\[
\langle a,b,c\,|\,ca-bab,\,cb-\omega baa,baab-\lambda baaa,\,babb-\mu
baaa,\,pa-ybaaa,\,pb-zbaaa,\,pc-tbaaa,\,\text{class }4\rangle .
\]

For each pair $\lambda ,\mu $ we compute the subgroup of the automorphism
group which fixes $\lambda ,\mu $, and compute its action on the parameters $%
y,z,\dot{t}$. It turns out that we need to treat the pair $\lambda =\mu =0$
separately from the other pairs.

If $\lambda =\mu =0$. Then the subgroup of the automorphism group we need to
consider maps $a,b,c$ to $a^{\prime },b^{\prime },c^{\prime }$ where 
\begin{eqnarray*}
a^{\prime } &=&\alpha a, \\
b^{\prime } &=&\pm \alpha b+\varepsilon c, \\
c^{\prime } &=&\alpha ^{2}c,
\end{eqnarray*}%
with $b^{\prime }a^{\prime }a^{\prime }a^{\prime }=\pm \alpha ^{4}baaa$.

In all other cases we can assume that if $pc\neq 0$ then $pa=pb=0$. A 
\textsc{Magma} program to compute a set of representatives for the
parameters $\lambda ,\mu ,y,z,t$ is given in notes6.173.m.

\end{document}
